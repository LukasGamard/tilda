\mode*

\section{Tentafrågor}
% Denna sektion tar ~35 minuter att gå igenom

\begin{frame}
  \begin{exercise}
    \begin{itemize}
      \item En miljon dumbolotter säljs var månad.
      \item För varje lott sparas lottnumret och köparen i ett objekt.
      \item En lista med en miljon objekt finns alltså i datorn vid dragningen, 
        då tusen vinstnummer slumpas fram, ett efter ett.
      \item För varje nummer måste hela listan letas igenom, eftersom den är 
        osorterad.
    \end{itemize}
    \begin{enumerate}
      \item Hur många jämförelser får man räkna med totalt? Lönar det sej att 
        först sortera listan?
    \end{enumerate}
  \end{exercise}
\end{frame}

\begin{frame}
  \begin{exercise}
    \begin{enumerate}
      \item Hur många jämförelser får man räkna med totalt? Lönar det sej att 
        först sortera listan?
    \end{enumerate}
  \end{exercise}

  \begin{solution}
    \begin{itemize}
      \item Ja. I en osorterad lista krävs cirka en halv miljon jämförelser för 
        varje sökning, d.v.s.\ totalt en halv miljard.
      \item \({\color{red}\frac{1}{2}}\cdot 1\,000\,000\) jämförelser för varje 
        vinstnummer, \(1000\) vinstnummer, totalt \(0.5\cdot 10^{9}\).
      \item Quicksort kräver \(O(N\log N)\) jämförelser.
      \item \(1\,000\,000\cdot \log_{\color{red}2} 1\,000\,000 = 10^6\cdot 
        6\log_2 
            10 \approx 18\cdot 10^6\)
      \item Varje binärsökning tar \(O(\log N)\).
      \item Detta ger \(\log_{\color{red}2} 10^6\approx 18\) per vinstnummer, 
        \(18\cdot 1000\) totalt.
    \end{itemize}
  \end{solution}
\end{frame}

\begin{frame}
  \begin{exercise}
    \begin{itemize}
      \item Höjdhoppsfederationens databas över världens alla 
        höjdhoppstävlingsresultat består av objekt med bland annat fälten 
        datum, plats, höjd (cm), hoppare och rivit/klarat.
      \item På skivminnet ligger objekten i datumordning, men man vill sortera 
        om dom i resultatordning, nämligen klarade hopp före rivna och höga 
        hopp före låga.
    \end{itemize}
    \begin{enumerate}
      \item Vilken sorteringsmetod är bäst? Motivera utförligt.
    \end{enumerate}
  \end{exercise}
\end{frame}

\begin{frame}
  \begin{exercise}
    \begin{enumerate}
      \item Vilken sorteringsmetod är bäst? Motivera utförligt.
    \end{enumerate}
  \end{exercise}

  \begin{solution}
    \begin{itemize}
      \item Väldigt många hopp totalt.
      \item Men varje hopp är bara mellan \si{100} och \SI{300}{\centi\metre}.
      \item Antalet hopp är många fler än möjliga hoppvärden, då är 
        distributionsräkning bästa sorteringsalgoritm.
      \item Rivit/klarat fördubblar antalet möjliga värden.
    \end{itemize}
  \end{solution}
\end{frame}

\begin{frame}
  \begin{solution}[Fortsättning, algoritmen]
    \begin{itemize}
      \item Läs igenom filen två gånger.
      \item Första gången för att räkna hur många hopp det finns av varje 
        rivit/klarat plus höjd.
      \item Avsätt lagom stort segment av listan för varje objekt.
      \item Vid andra genomläsningen av filen kan varje hopp sättas in på rätt 
        ställe i listan.
    \end{itemize}
  \end{solution}
\end{frame}

\begin{frame}
  \begin{exercise}
    \begin{itemize}
      \item För att kontrollera sanningen i detta talesätt har man i en fil 
        samlat tre miljoner datum för svenska julgranars utkastning.
      \item Man vill veta mediandatum, alltså det datum då hälften av granarna 
        slängts ut, ut, ut och hälften ännu står gröna och granna i stugan.
    \end{itemize}
    \begin{enumerate}
      \item Rangordna följande sex föreslagna metoder efter deras effektivitet:
        \begin{itemize}
          \item binärsökning,
          \item hashning,
          \item insättningssortering,
          \item distributionsräkning,
          \item djupet-först-sökning,
          \item trappsortering (heap sort).
        \end{itemize}
    \end{enumerate}
  \end{exercise}
\end{frame}

\begin{frame}
  \begin{remark}
    \begin{itemize}
      \item Det finns 365 olika datum på året.
      \item Vi har tre miljoner datum för när julgranen åkte ut.
    \end{itemize}
  \end{remark}
  \begin{solution}
    \begin{itemize}
      \item Observationen ovan gör att distributionsräkning är bäst, \(O(N)\).
      \item Trappsortering (heap sort) är näst bäst, \(O(N\log N)\), då man kan 
        avbryta när hälften är sorterade.
      \item Insättningssortering funkar också, \(O(N^2)\).
      \item Hashning går att använda om hashfunktionen inte ger krockar.
      \item Binärsökning och djupet-först-sökning går inte att använda för att 
        hitta medianen.
    \end{itemize}
  \end{solution}
\end{frame}

\begin{frame}
  \begin{exercise}
    \begin{itemize}
      \item Riksskatteverkets databas med nio miljoner svenskar finns sorterad 
        på efternamn.
      \item Man vill sortera om den på personnummer.
    \end{itemize}
    \begin{enumerate}
      \item Hur många jämförelser krävs med quicksort?
      \item Hur många med den bästa metoden?
    \end{enumerate}
  \end{exercise}
\end{frame}

\begin{frame}
  \begin{exercise}
    \begin{enumerate}
      \item Hur många jämförelser krävs med quicksort?
      \item Hur många med den bästa metoden?
    \end{enumerate}
  \end{exercise}

  \begin{solution}
    \begin{itemize}
      \item Quicksort (\(O(N\log N)\)).
      \item \(N = 9\cdot 10^6\approx 2^{23}\implies 9\cdot 10^6\cdot 23 = 
        207\cdot 10^6\).
      \item Radixsortering är snabbare.
    \end{itemize}
  \end{solution}
\end{frame}

\begin{frame}
  \begin{definition}[Radixsortering]
    \begin{itemize}
      \item Räkna ut längden på längsta talet, \(k\).
      \item Använd räknesortering för att sortera efter sista siffran i talen.
      \item Använd sedan räknesortering för att sortera efter näst sista.
      \item Fortsätt tills alla siffror (\(k\) stycken) är genomgångna.
    \end{itemize}
  \end{definition}

  \begin{solution}[Fortsättning, radixsortering]
    \begin{itemize}
      \item Gå igenom alla personnummer (\(k = 10\) siffror).
      \item Dela upp i tio buntar efter sista siffran, lägg ihop.
      \item Dela upp i tio nya buntar efter näst sista siffran, lägg ihop.
      \item \dots
      \item Detta ger \(10\cdot 9\cdot 10^6 = 90\cdot 10^6\).
    \end{itemize}
  \end{solution}
\end{frame}

\mode<all>{\endinput}

\begin{frame}[fragile]
  \begin{exercise}
    \begin{itemize}
      \item Nu när det blir mörkare om kvällarna känns det extra viktigt att ha 
        lysen till cykeln.
      \item Du har lagt in lamporna med priset som prioritet i en min-heap.
      \item På vektorform ser heapen ut så här:
        \begin{minted}{text}
          10 40 30 42 41 48 50 49
        \end{minted}
    \end{itemize}
    \begin{enumerate}
      \item Rita upp heapen på trädform och visa sen hur det ser ut när man 
        plockar ut två element (du vill ju inte köpa de allra billigaste).
        Visa minst fem steg.
      \item Skriv slutligen upp heapen på vektorform igen.
    \end{enumerate}
  \end{exercise}
\end{frame}
